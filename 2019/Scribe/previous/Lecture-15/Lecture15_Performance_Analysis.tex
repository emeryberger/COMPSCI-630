\documentclass[twoside]{article}
\setlength{\oddsidemargin}{0.25 in}
\setlength{\evensidemargin}{-0.25 in}
\setlength{\topmargin}{-0.6 in}
\setlength{\textwidth}{6.5 in}
\setlength{\textheight}{8.5 in}
\setlength{\headsep}{0.75 in}
\setlength{\parindent}{0 in}
\setlength{\parskip}{0.1 in}

\usepackage{graphicx}
\usepackage{url}

%
% The following commands sets up the lecnum (lecture number)
% counter and make various numbering schemes work relative
% to the lecture number.
%
\newcounter{lecnum}
\renewcommand{\thepage}{\thelecnum-\arabic{page}}
\renewcommand{\thesection}{\thelecnum.\arabic{section}}
\renewcommand{\theequation}{\thelecnum.\arabic{equation}}
\renewcommand{\thefigure}{\thelecnum.\arabic{figure}}
\renewcommand{\thetable}{\thelecnum.\arabic{table}}
\newcommand{\dnl}{\mbox{}\par}

%
% The following macro is used to generate the header.
%
\newcommand{\lecture}[4]{
  \pagestyle{myheadings}
  \thispagestyle{plain}
  \newpage
  \setcounter{lecnum}{#1}
  \setcounter{page}{1}
  \noindent
  \begin{center}
  \framebox{
     \vbox{\vspace{2mm}
   \hbox to 6.28in { {\bf CMPSCI~377~~~Operating Systems
                       \hfill Fall 2009} }
      \vspace{4mm}
      \hbox to 6.28in { {\Large \hfill Lecture #1  \hfill} }
%       \hbox to 6.28in { {\Large \hfill Lecture #1: #2  \hfill} }
      \vspace{2mm}
      \hbox to 6.28in { {\it Lecturer: #3 \hfill Scribe: #4} }
     \vspace{2mm}}
  }
  \end{center}
  \markboth{Lecture #1: #2}{Lecture #1: #2}
  \vspace*{4mm}
}

%
% Convention for citations is authors' initials followed by the year.
% For example, to cite a paper by Leighton and Maggs you would type
% \cite{LM89}, and to cite a paper by Strassen you would type \cite{S69}.
% (To avoid bibliography problems, for now we redefine the \cite command.)
%
\renewcommand{\cite}[1]{[#1]}

% \input{epsf}

%Use this command for a figure; it puts a figure in wherever you want it.
%usage: \fig{NUMBER}{FIGURE-SIZE}{CAPTION}{FILENAME}
\newcommand{\fig}[4]{
           \vspace{0.2 in}
           \setlength{\epsfxsize}{#2}
           \centerline{\epsfbox{#4}}
           \begin{center}
           Figure \thelecnum.#1:~#3
           \end{center}
   }

% Use these for theorems, lemmas, proofs, etc.
\newtheorem{theorem}{Theorem}[lecnum]
\newtheorem{lemma}[theorem]{Lemma}
\newtheorem{proposition}[theorem]{Proposition}
\newtheorem{claim}[theorem]{Claim}
\newtheorem{corollary}[theorem]{Corollary}
\newtheorem{definition}[theorem]{Definition}
\newenvironment{proof}{{\bf Proof:}}{\hfill\rule{2mm}{2mm}}

% Some useful equation alignment commands, borrowed from TeX
\makeatletter
\def\eqalign#1{\,\vcenter{\openup\jot\m@th
 \ialign{\strut\hfil$\displaystyle{##}$&$\displaystyle{{}##}$\hfil
     \crcr#1\crcr}}\,}
\def\eqalignno#1{\displ@y \tabskip\@centering
 \halign to\displaywidth{\hfil$\displaystyle{##}$\tabskip\z@skip
   &$\displaystyle{{}##}$\hfil\tabskip\@centering
   &\llap{$##$}\tabskip\z@skip\crcr
   #1\crcr}}
\def\leqalignno#1{\displ@y \tabskip\@centering
 \halign to\displaywidth{\hfil$\displaystyle{##}$\tabskip\z@skip
   &$\displaystyle{{}##}$\hfil\tabskip\@centering
   &\kern-\displaywidth\rlap{$##$}\tabskip\displaywidth\crcr
   #1\crcr}}
\makeatother

% **** IF YOU WANT TO DEFINE ADDITIONAL MACROS FOR YOURSELF, PUT THEM HERE:



% Some general latex examples and examples making use of the
% macros follow.

\begin{document}

%FILL IN THE RIGHT INFO.
%\lecture{**LECTURE-NUMBER**}{**DATE**}{**LECTURER**}{**SCRIBE**}
\lecture{15}{November 4}{Emery Berger}{Theodore Sudol, ???}

\section{Performance Analysis}

From the beginning of a project, decisions like language choice affect
performance. For example, Ruby is an order of magnitude slower than C, C++ or
Java. There is only so much performance you can squeeze out of it simply due to
the language design.

\subsection{Programming Language Concurrency}

\textbf{C:} pthread\_create(\ldots) uses real OS threads.

\textbf{Java:} Originally used ``GreenThreads'', where multiple JVM threads were
mapped to one shared OS thread; this has concurrency (overlapping communication
and computation) but is not parallel. Normal Java threads are $1 : 1$, or $m :
m$, where each Java thread is mapped to a corresponding OS thread. Greenthreads
were $m : 1$.

\textbf{Fibers} are lightweight threads that can be described as $m : n, m < n$,
i.e. multiple fibers per OS thread.

The $m : 1$ scheme has some advantages. It offers simpler and cheaper
synchronization, fine-grained control of the scheduler, and no context switches,
which improves performance. Typically, each thread has its own large stack that
is bounded by memory-protected pages. Thanks to the finer control over threads,
the programming language can use smaller stacks and save memory with the $m : 1$
model. The $m : n$ model conveys the same advantage.

\textbf{Work Queues} are used to implement the $m : n$ threading model. Each OS
thread has a queue of thread states defined by the programming language. Each PL
thread is processed as the OS threads work through their queues. To ensure
performance, the work queues are load balanced (work is spread evenly if
possible) and implement randomized load stealing: if an OS thread's work queue
empties, it takes a thread from the back of another OS thread's queue.

\subsection{Benchmarking}

A program's performance is very complex: One must know the runtime environment,
along with the thread model and the performance model of the language. Most
languages do not have these models.

\textbf{Cilk} is a programming language with the first formalized conception of
randomized work stealing and a formal performance model: $T_p = O(\frac{T_1}{p})
+ T_\infty$. $T_p$ is the execution time with $p$ threads, $T_1$ is the
execution time with a single thread, and $T_\infty$ is the execution time of the
serial portion of the program. $T_1$ is determined by simply running the
program, and $T_\infty$ can be determined by the program itself. This
performance model bears a strong resemblance to Amdahl's law.

\textbf{gprof} is a benchmarking tool that profiles each function in the
program.

\subsection{Handling Bias}

\textbf{Measurement bias} occurs when a system used to measure two or more items
favors one over the others. This skews measurements and can lead to incorrect
results.

\textbf{Regression analysis} measures the correlation between two variables.
However, one must be aware of \emph{latent confounding variables} that can
affect the variables under analysis. Measurement bias manifests as a latent
confounding variable, for example.

\textbf{Hypothesis Testing:} Start with the \emph{Null Hypothesis:} Nothing
happened. For example, given two programs $A$ and $A^\prime$, where the latter
is optimized. The null hypothesis is that the programs have the same
performance.

\begin{enumerate}
  \item Assume the null hypothesis: $T_A = T_{A^\prime}$
  \item Run the experiment to \emph{reject} the null hypothesis
  \item Define a significance level or confidence interval: $p \le 0.05$
  \item If the experiment shows that $T_A$ and $T_{A^\prime}$ differ
    significantly to confidence $\le 0.05$, then reject the null hypothesis.
\end{enumerate}

For example, say we are testing a new drug called ``Snkoil''. The null
hypothesis is that Snkoil performs no better than the placebo. This is tested in
a double blind, randomized trial: Subjects in each group (experimental and
control) are randomized, and the people running the experiment should not know
which group each subject is in.

But what happens if Snkoil tastes bad? Because we associate bad tasting things
with medicine, the subjects may realize they're being treated and the placebo
effect kicks in. This pierces the double blind and invalidates the experiment.

In computer science, confounding variables are very well hidden: environment
variables, user names, the current directory, etc. can all change the alignment
of code in memory. This affects performance by causing misses with the various
hash tables in the computer, namely the caches, TLB and branch predictor.

Running an experiment with every possible configuration of known confounding
variables would eliminate the measurement bias they cause, but this would
require several hundred factorial runs.

\textbf{Stabilizer} gives a practical way to remove the influence of confounding
variables. It essentially turns the experiment into a single blind randomized
study. (Not double blind, because the computer doesn't care what you think.)

\end{document}
